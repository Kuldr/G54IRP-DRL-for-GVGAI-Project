\documentclass[a4paper]{article}
\usepackage{lipsum}
\usepackage{url}
\usepackage{graphicx}
\usepackage{listings}
\lstset{language=Python}
\usepackage[margin=2cm]{geometry}
\usepackage{indentfirst}
\usepackage{xcolor}
\usepackage{lmodern}
\usepackage{enumitem}
\renewcommand{\familydefault}{\sfdefault}
\graphicspath{ {images/} }

\begin{document}

% TODO: Anything else needed for titlepage
\begin{titlepage}
	\newcommand{\HRule}{\rule{\linewidth}{0.5mm}}
	\center{}

    %Headings
	\LARGE{University of Nottingham}\\[1.5cm]
	\Large{Computer Science with Artificial Intelligence MSci}\\[0.5cm]
	\large{G54IRP/COMP4027 - Individual Research Project}\\[0.5cm]

    %Title
	\HRule{}\\[0.4cm]
	{\huge\bfseries AI for General Video Game Playing}\\[0.4cm]
	\HRule{}\\[1.5cm]

    %Authors
	\begin{minipage}{0.4\textwidth}
		\begin{flushleft}
			\large
			\textit{Author}\\
			Benjamin Charlton\\
            psybc3@nottingham.ac.uk\\
            4262648
		\end{flushleft}
	\end{minipage}
    \begin{minipage}{0.4\textwidth}
		\begin{flushright}
			\large
			\textit{Supervisor}\\
			Dr.\@ Ender \"Ozcan\\
            Ender.Ozcan@nottingham.ac.uk
		\end{flushright}
	\end{minipage}

    %Date
	\vfill\vfill\vfill
	{\large7\textsuperscript{th} December 2017}
	\vfill

\end{titlepage}

%Contents Page
\pagenumbering{roman}
\tableofcontents
\pagebreak

\pagenumbering{arabic}
\section{Introduction}
\subsection{Introduction}
\begin{itemize}
    \item Popularity of Video Games
    \item Variety of Video games being played
    \item Video Games as a testing ground for AI
\end{itemize}
\subsection{Planning VS learning}
\begin{itemize}
    \item As the title says, just to get some definitions out of the way
\end{itemize}
\subsection{Motivation}
\begin{itemize}
    \item Individual desire to get better at deep reinforcement learning
    \item Learning more Methods
    \item Applying Methods
    \item Learning to use more robust and real world frameworks for DRL \\
    \item Seeing the state of the art in DRL and AI game playing
\end{itemize}
\subsection{Aims and Objectives}
\begin{itemize}
    \item Look at the project plan aims and objectives
\end{itemize}

\section{Related Work}
\subsection{AI and Game Playing}
\begin{itemize}
    \item AI approaches have specialised heuristics or only been developed for a single game
    \\
    \item Go - previous interim report / PP
    \item Chess - previous interim report / PP
    \item OpenAIFive
    \item GET MORE THINGS FROM PP
    \\
    \item Dartmouth Workshop
\end{itemize}

\subsubsection{Early Artificial Intelligence}
%TODO: Maybe extend / re sort this section
The history of AI game playing begins near the start of artificial intelligence as a field, in the 1950s.
Strachey created a draughts player for one of the first general computing machines (Manchester Ferranti Mark I) which by  1952 could ``play a complete game of draughts at a reasonable speed''\cite{BreifHistoryComputing}.
Prinz wrote a simplified chess player for the Manchester Machine as well which could solve the mate-in-two problem.
This meant that if there was a checkmate solution in 2 turns it could successfully find it\cite{BreifHistoryComputing}.
Prinz simply used an exhaustive search technique to find the correct moves, and even though computing power was limited at the time it was clear that this wouldn't scale to full games.
This lead to Turning starting to program `Turbochamp' a chess program that would be able to play a full game of chess using heuristics\cite{BreifHistoryComputing}
\par
These simple games were made before the term artificial intelligence was being used even in an academic setting showing how natural AI and game playing go together.
\subsection{GVGAI and VGDL}
\begin{itemize}
    \item TODO Make some notes up here
\end{itemize}
\subsection{OPEN AI GYM and GVGAI GYM}
\begin{itemize}
    \item What is OPEN AI GYM
    \item Why is OPEN AI GYM helpful for RL
    \item GVGAI GYM
    \item Initial results from GVG AI GYM paper
    \item Maybe some more here :)
\end{itemize}
\cite{GVGAIGYM}
\subsection{Subsection to end all subsections}
\begin{itemize}
    \item Maybe something about the open AI baselines
    \item Depends what technique I would be using but I should look into that
    \item Where does reinforcement learning come into this bad buoy
\end{itemize}


\pagebreak
\section{Appendix}

\subsection{Meeting Minutes}


\pagebreak
\subsection{Work Plan}
% \begin{center}
    % \includegraphics[height=24.8cm]{workPlan.png}
% \end{center}


%Bibliography
\subsection{References}
\bibliography{InterimReport}
\bibliographystyle{plain}


\end{document}
